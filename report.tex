
% report.tex
%by 
%alemedeiros (alexandre.medeiros@students.ic.unicamp.br)
%kuroikizu (lucas.tadeuteixeira@gmail.com)
%

\documentclass[11pt,a4paper,titlepage]{article}
\usepackage{fullpage}
\usepackage[brazilian]{babel}
\usepackage[utf8]{inputenc}
\usepackage[T1]{fontenc}
\usepackage{indentfirst}
\usepackage{lmodern}
% code listing
\usepackage{listings}
% algorithms
\usepackage[noend]{algpseudocode}
\usepackage{algorithm}
% mathematical tools
\usepackage{mathtools}
%table tools
\usepackage{csvsimple}
\usepackage{longtable}
%correct bad floats
\usepackage{placeins}
\usepackage{appendix}
% already includes titling and graphicx packages
% cover.tex
%   by alemedeiros (alexandre.medeiros@students.ic.unicamp.br)
%
% cover for mc458 (design and analysis of algorithm) assignment reports

\usepackage{graphicx}
\usepackage{titling}
\input{aux/abaco}
\DeclareGraphicsExtensions{.pdf,.png,.jpg}

% define o estilo da capa gerada pelo \maketitle
% título
\pretitle{ % pré-titulo
  \noindent
  \begin{minipage}[c]{.2\textwidth}
    \ABACO{1}{9}{6}{9}{1}
  \end{minipage}
  \begin{minipage}[c]{.6\textwidth}
    \begin{center}
      {\huge \sc Instituto de Computação}\\[20pt]
      {\Large \sc Universidade Estadual de Campinas}
    \end{center}
  \end{minipage}
  \begin{minipage}[c]{.2\textwidth}
    \includegraphics[width=\textwidth]{imgs/logo-unicamp}
  \end{minipage}
\vskip 120pt\begin{center}\LARGE \bf}

\posttitle{ % pós-titulo
  \par\end{center}\vskip 20pt\par
  \begin{center}
    \large MC458 -- Projeto e Análise de Algoritmos
  \end{center}
\vskip 70pt}

% autor
\preauthor{\begin{center}\vskip 15pt\par
\large \lineskip 0.5em%
\begin{tabular}[t]{c}}
\postauthor{\end{tabular}\par\end{center}}
% data
\predate{\vfill\begin{center}\large}
\postdate{\par\end{center}}


% information about author and title
\title{Modelagem de consumo de energia em dispositivos moveis, com enfase na plataforma Exynos 4412 - Samsung Galaxy S3}
\author{Aluno: Gabriel Hidasy Rezende RA 116928 \and Orientador: Prof. Dr. Sandro Rigo}
\date{Campinas, \today}

% references become hyperlinks and adds metainfo to the pdf file
\usepackage{hyperref}
\hypersetup{
  bookmarks=true,
  pdftitle={\thetitle},
  pdfauthor={\theauthor},
  hidelinks
}

% style for C source listings
\lstset{
  language=C,
  %escapeinside={(*}{*)}, % use to escape LaTeX commands
  breaklines,
  breakatwhitespace,
  frame=lines,
  framerule=0.3mm,
  numbers=left,
  numberstyle=\tiny,
  basicstyle=\small,
  float=h!
}

\renewcommand\lstlistlistingname{Programas}
\renewcommand\lstlistingname{Programa}

% pseudo-code config
\floatname{algorithm}{Algoritmo}
\renewcommand\listalgorithmname{Algoritmos}
\algrenewcommand\alglinenumber[1]{{\footnotesize#1}}


\begin{document}
\pagenumbering{roman}
\maketitle

\tableofcontents
\newpage

\pagenumbering{arabic}
\begin{center}
\section*{Resumo}
\end{center}
\paragraph{} O projeto consiste na modelagem do consumo de energia de dispositivos moveis ARM, mais especificamente do Samsung Galaxy S3, o trabalho completo foge do escopo deste relatório, serão detalhadas as atividades desenvolvidas ao longo do segundo semestre de 2013.
\paragraph{} Dentre os assuntos tratados terá foco a montagem da infraestrutura para os experimentos a ser feitos no decorrer do projeto, especialmente sobre os desafios da medição da energia do dispositivo.
\paragraph{} Assim sera relatado como foi medida a energia consumida pelo aparelho, quais os equipamentos utilizados e particularidades da plataforma que tiveram de ser contornadas, além de informações sobre os benchmarks desenvolvidos e planejados e as metodologias a ser avaliadas para modelar o gasto do consumo de cada dispositivo.
\paragraph{} O restante desse documento está dividido da seguinte forma: A seção 1 da uma introdução ao problema, a seção 2 detalha o cronograma planejado, a seção 3 a infraestrutura montada para a coleta de dados de potencia do aparelho, a seção 4 a infraestrutura de coleta dos contadores de desempenho do aparelho, a seção 5 trata da sincronização entre os processos de medição, a seção 6 dá detalhes dos benchmarks utilizados e por fim a seção 7 descreve a metodologia em uso no final do semestre (22/11/13)
\newpage
\section{Introdução}
\paragraph{} Como as atividades estão sendo realizadas em equipe, nem todas as tarefas descritas foram realizadas por todos os participantes do projeto, por isso o enfoque deste relatório será nas tarefas que realizei.
\paragraph{} Nos últimos anos uma tendencia observada foi o aumento exponencial nas funções desempenhadas por celulares, com a criação da categoria dos smartphones, essa escalada de usos foi ao mesmo tempo causa e efeito de um aumento substancial no poder de processamento desses dispositivos, além de influenciar no formato dos aparelhos (telas maiores, touchscreen, espessura cada vez menor levando a baterias mais finas) e a inclusão de outros componentes agora comuns (GPS, Wifi, redes moveis de alta velocidade como 3G e 4G, placas gráficas e grandes quantidades de memoria e armazenamento).
\paragraph{} O resultado disso foi um grande aumento do consumo de energia por esses dispositivos, porém não se verificou um aumento condizente na capacidade das baterias, por isso se antes celulares tinham carga para semanas agora dificilmente se encontra um aparelho que não precise ser recarregado diariamente.
\subsection{Objetivos}
\paragraph{} A busca de maneiras de otimizar e reduzir o consumo de energia exige uma caracterização do problema e esse é o objetivo dos trabalho desenvolvidos nesse projeto, em particular pretendemos 
\subparagraph{•} Medir o consumo de energia no processador do aparelho sob diversas cargas, de modo a caracterizar o consumo de energia baseado em contadores de desempenho
\subparagraph{•} Avaliar o quão relevante é a tela do aparelho no consumo total do mesmo, em particular o efeito das configurações de brilho da mesma.
\subparagraph{•} Caracterizar o consumo de energia de cada um dos diversos transmissores do aparelho sob diversos modos de operação (variando a qualidade do sinal e a velocidade com que os dados são transmitidos).
\paragraph{} A partir dos modelos criados será possível prever o consumo de energia de uma aplicação e possivelmente encontrar maneiras de otimizar as mesmas, e em particular das camadas inferiores do sistema do celular (a maquina virtual Dalvik e a Bionic, substituta da glibc, no caso do Android), permitindo uma maior autonomia para os aparelhos atuais.

\section{Cronograma}
\paragraph{} O cronograma completo das atividades, com duração de dois anos, está includo no Anexo 1
\subsection{Atividades sendo desenvolvidas}
\paragraph{4 - Construção de infraestrutura} Analisar alternativas para medida do 
consumo de energia, temos um esquema funcional implementado porém é necessária a criação de um mockup da bateria para garantir a estabilidade do conjunto 
\paragraph{5 - Medição de contadores de desempenho} Temos um programa capaz de medir contadores de desempenho porém ainda não sabemos quais contadores podem ser medidos em paralelo nem quais contadores vão de fato ser importantes para o resultado final
\paragraph{6 - Seleção de benchmarks} Decidimos implementar benchmarks para termos maior controle sobre a duração e carga de cada um, isolando assim benchmarks de CPU, GPU, tela, rede e GPS, os benchmarks de tela e de processador já foram planejados e implementados, os de rede já foram planejados, ainda precisamos determinar como estressar o GPS e a GPU
\paragraph{} Vale ressaltar que já estamos trabalhando com o córtex A15 da freescale em paralelo com o da Samsung, para testar a generalidade da metodologia, e já temos um programa para a leitura da temperatura do processador (que como quase todas as medidas podem ser encontradas lendo um arquivo, /sys/devices/platform/s5p-tmu/temperature)
\section{Infraestrutura energética}
\subsection{Medida de energia} 
\paragraph{} Precisamos para o projeto de uma maneira de medir com precisão e gravar qual o consumo instantâneo do celular em alta frequência devido a rápida variação do consumo de energia do aparelho, para isso inserimos uma resistência em série com os conectores de energia do aparelho, de $0,25\Omega$ e utilizamos um DAC (NI-6212USB da National Instruments) para medir a tensão entre os terminais da resistência, a partir disso através da lei de Coulomb ($I=U/R$) podemos obter a corrente sendo consumida pelo celular, sabendo a corrente consumida pelo celular e a tensão dos terminais que o alimentam (igual a tensão da bateria original menos a tensão na resistência) podemos determinar a energia sendo gasta pelo mesmo pela relação $P=U*I$.
\paragraph{} A abordagem apresentou alguns problemas que tiveram que ser tratados:
\subparagraph*{1º} A própria ligação da resistência exigiu a criação de uma bateria mockup (uma bateria falsa que permite que se ligue alimentação externa) porque o celular infelizmente só pode ser alimentado pelos plugues da bateria que são protegidos demais para ligarmos a resistência diretamente.
\subparagraph*{2º} A resistência elétrica total do conjunto de mockup e resistência de medição, mais os fios utilizados tornaram impossível para a bateria do celular sustentar o mesmo em picos de consumo, o que levava o mesmo a desligar durante os benchmarks, para resolver o problema trocamos a bateria por uma fonte de precisão, o que além disso torna o experimento menos dependente de variações da carga e da saúde da bateria.
\subparagraph*{3º} O celular é um circuito de altíssima frequência (até 1,6GHz) e o consumo de energia do mesmo varia enormemente (medimos entre 0.01A em momentos de inatividade a mais de 1.4A capturando uma imagem com o uso do Flash), e existem circuitos reguladores de energia dentro do aparelho, assim os dados coletados mostram enormes variações de consumo em alta frequência, assim precisamos suavizar os resultados aumentando a frequência da medição e tomando a media dos resultados em um intervalo de tempo maior
\subsection{Montagem final}
\paragraph{} Terminamos por comprar uma fonte Minipa MPL-3305M que configuramos para manter uma saída constante de $5v$, cortamos o fio de alimentação negativo e soldamos a resistência nele de modo que a mesma fica em serie com o celular, os terminais então são ligados a bateria mockup que alimenta o celular, outro par de cabos é ligado nas entradas AI0 do DAC e em paralelo com a resistência, de modo que o DAC lê a tensão no resistor, no computador temos um programa que lê os dados do DAC (NI-6121USB), também é possível usar o LabView para coletar dados.
\paragraph{} Os dados de energia coletados tem que ser adequados através de uma trendline devido a natureza do circuito do celular, que varia a corrente requerida da bateria extremamente rápido.
\subsection{Fotos}
(TODO) Inserir fotos da montagem final e um diagrama do circuito
\section{Contadores de desempenho}
\subsection{Processador}
\paragraph{} Dado que este é um componente complexo demais para ser avaliado apenas pelo seu tempo de atividade e frequência optamos pela construção de um programa para medir em intervalos periódicos os contadores de desempenho internos do processador e do sistema (instruções executadas, saltos, cache-hits e cache-misses, pagefaults...), a plataforma utilizado (O exynos 4412) possui 6 contadores de hardware de uso geral e alguns contadores fixos (como o numero de ciclos que o processador executou), os problemas encontrados nessa seção foram
\subparagraph{1º} O kernel padrão do Android não vem com o perf\_events (que da acesso a esses contadores) habilitado, ele teve que ser recompilado para dar suporte a essa interface.
\subparagraph{2º} Medir esses contadores com o método padrão, o perf, não é suficiente pois o mesmo só conta a execução completa do programa, precisamos de medições granulares
\subparagraph{3º} Tentamos então usar a libpfm que dá acesso aos mesmos contadores, foi preciso uma grande quantidade de testes até obtermos um programa funcional no kernel do Android
\subparagraph{4º} O processador não tem capacidade de medir todos os eventos que acontecem com o mesmo ao mesmo tempo, por isso precisamos escolher um conjunto de contadores de hardware representativo. TODO quais foram
\subsection{Tela}
\paragraph{} A tela pode ser representada por basicamente pelo estado ligada-desligada e pelo nível de brilho da mesma no estado ligada, o diferença no consumo da mesma em imagens estáticas ou em imagens dinâmicas é praticamente nula.
\subsection{Redes}
\paragraph{} Cada transmissor deve ser caracterizado separadamente, temos disponíveis redes 2,5G, 3G e Wifi, será bastante complicado caracterizar as redes móveis pois o consumo delas varia com a distancia da estação base e do numero de usuários, podemos tentar estabelecer uma relação entre a velocidade da transferência e consumo tanto para download quanto para upload, para o Wifi ainda podemos medir o efeito da distancia entre o celular e o access point (que é proporcional a qualidade do sinal) 
\subsection{GPS}
\paragraph{} O consumo do GPS pode ser modelado pelo custo de obter a localização e pelo numero de vezes que a aplicação pede que a mesma seja atualizada por unidade de tempo, é possível que condições atmosféricas e grandes deslocamentos sem atualizações na localização levem a requisição a demorar mais e consumir mais energia. 
\subsection{GPU}
\paragraph{} Ainda não temos certeza sobre se é possível, e se for como, medir os contadores de desempenho da GPU como os do processador, o componente é complexo demais para ser caracterizado apenas pelo estado ligado/desligado.
\section{Sincronização}
\paragraph{} A sincronização é um problema chave desse projeto, para obtermos correlações confiáveis sobre quais contadores de desempenho tem maior impacto sobre a duração da bateria, o problema pode até ser reduzido rodando benchmarks longos em relação a defeitos de sincronia mas devemos buscar uma boa aproximação
\paragraph{} A sincronização também será crucial ao chegarmos no final do cronograma, será necessário correlacionar energia e temperatura para permitir a implementação de técnicas de "computational sprinting", um modelo defeituoso nessa área poderia resultar em desempenho abaixo do máximo se superestimássemos o aumento de temperatura, e se a subestimássemos a falha poderia ser catastrófica caso não fosse detectada pelo controlador de temperatura do chip e no minimo levaria a longos períodos de "cooldown" onde o chip seria obrigado a trabalhar em modo de economia de energia após  superaquecer, além da provável diminuição da vida útil do processador que iria pelo estresse decorrente dos superaquecimentos frequentes.
\paragraph{} Precisamos de uma maneira de sincronizar as medidas, a maneira mais simples de fazer isso que encontramos é o flash da  do aparelho, basicamente ao iniciar os benchmarks e contadores damos um lampejo do flash que aparece como um pico nos dados de energia e é simples de sincronizar com os contadores de desempenho pois pode ser ser ativado de dentro do próprio programa que mede a energia (basta escrever 1 em um arquivo para ligar o flash e 0 para desligar), além disso pretendemos sincronizar os relógios de todos os aparelhos com a tecnologia do NTP em um servidor local, (o protocolo permite sincronização dos relógios na casa dos milissegundos em redes confiáveis com atrasos pequenos e constantes, assim poderemos incluir timestamps relativamente confiáveis nos testes.

\section{Benchmarks} 
\paragraph{} Os benchmarks devem durar um tempo razoavelmente longo para termos resultados confiáveis, visto que o celular pode apresentar variações de consumo devido a processos automáticos do Android e do kernel, e para minimizar o erro decorrente da medição, além de minimizar o problema que uma pequena falta de sincronização traria, (se por ventura começarmos a medir os dados com 1 decimo de segundo de atraso em medidas de 1 segundo teremos um problema, se as medidas durarem 10 minutos a diferença será quase indetectável).


\subsection{Processador}
\paragraph{} Como base escolhemos a multiplicação de matrizes, por ser uma tarefa que pode durar o tempo que for necessário bastando para isso aumentar as matrizes ou repetir a tarefa varias vezes, além de ser facilmente paralelizável e permitir que, repetindo varias vezes matrizes pequenas tenhamos a computação toda em cache, ou com matrizes grandes um grande numero de cache-misses, usando exaustivamente a unidade de inteiros ou a de ponto flutuante, permitindo assim caracterizar varias situações de uso diferentes.

\subsection{Tela}
\paragraph{} Conseguimos controlar o brilho da tela a partir de arquivos de sistema no diretório /sys/ (na verdade são interfaces para estruturas internas do kernel, a localização exata varia entre as versões do Android), será feita também uma aplicação java para a tarefa, no futuro a aplicação poderá ser expandida para disparar outros benchmarks e fazer a coleta de dados.
\subsection{Redes}
\paragraph{} Usando o netcat podemos facilmente montar uma aplicação cliente servidor, e inclusive limitar a velocidade máxima da transferência (utilizando o comando pv -L) para verificar a influencia da mesma, gerenciar qual rede é usada é relativamente simples, o Wifi é facilmente controlado e o 3G pode ser desativado, porem não podemos garantir a qualidade do sinal móvel.
\subsection{GPS}
\paragraph{} Ainda não começamos a estudar como fazer um benchmark do GPS, a API permite que se requisite a localização GPS de tempos em tempos e podemos medir quanto tempo o receptor ficou ligado pelo tempo até a resposta, mas ainda não planejamos como coletar esses dados, isso será estudado mais a frente quando as tarefas principais estiverem concluídas.
\subsection{GPU}
\paragraph{} Ainda não começamos a estudar como fazer o benchmark de GPU, nem se é possível usar a GPU do aparelho sem passar pela maquina virtual Dalvik e se é possível ler os performance counters da mesma, isso será estudado mais a frente quando as tarefas principais estiverem concluídas.

\section{Metodologia}
\paragraph{} Criamos um script que dispara uma serie de eventos (benchmarks) e registra inicio e fim deles em uma linha do tempo, em outra linha do tempo pretendemos monitorar os performance counters do sistema e em outra o consumo de energia do celular, as três linhas do tempo terão de ser combinadas e repartidas entre eventos para análise, onde iremos cruzar dados a respeito do gasto de energia de cada atividade, cada contador de desempenho e cada evento ( como começar um download por wifi, ligar o 3G, mudanças no brilho da tela). 
\paragraph{} Nosso objetivo é a criação de um modelo de consumo baseado no tipo de processamento realizado e em que periféricos extras estiverem envolvidos, de modo que o consumo do aparelho seja previsível dada a aplicação.
\section{Anexos}
\subsection{Anexo I - Cronograma completo}
\paragraph{} O cronograma está dividido em 16 tarefas:\\
\subparagraph{1.} Revisão bibliográfica do estado da arte. \\
\subparagraph{2.} Seleção de estudantes e alocação. \\
\subparagraph{3.} Construção da infraestrutura – Parte I: estudar as características do Galaxy SIII e configurações. \\
\subparagraph{4.} Construção da infraestrutura – Parte II: analisar alternativas para medida do consumo de energia usando o sistema Galaxy SIII.\\ 
\subparagraph{5.} Arquitetura ARM: investigar como medir os contadores de desempenho e como controlar a frequência de operação dos processadores com arquitetura ARM. \\
\subparagraph{6.} Seleção de benchmarks: Analisar os benchmarks existentes para Android e escolher um subconjunto para nossa caracterização. A ideia é selecionar benchmarks que representam diferentes perfis de utilização, como jogos, navegação, produtividade, etc. \\
\subparagraph{7.} Caracterização de desempenho: avaliar a correlação entre o consumo de energia e os contadores de desempenho utilizando os benchmarks selecionados na plataforma Galaxy SIII. \\
\subparagraph{8.} Caracterização do consumo de energia – Phase I: O estudante utilizará a infraestrutura de medida de energia, definida no passo 4, para coletar medidas de consumo de energia e desempenho do sistema Galaxy SIII (Processador, GPU, memória, tela). Para o processador, este passo envolve a utilização de contadores de desempenho como proxy para o consumo de energia. As coletas de medidas devem ser realizadas para diferentes frequências de operação do processador. \\
\subparagraph{9.} Caracterização do consumo de energia – Phase II: Determinar quais parâmetros possuem correlação com o consumo de energia de outros componentes, como GPU, memória e a tela. \\
\subparagraph{10.} Caracterização do consumo de energia – Phase III: gerar modelos de consumo de energia utilizando a metodologia baseada em aprendizado de máquina e considerando diversas frequências de operação do processador ARM. Gerar modelos de consumo de energia para a GPU, memória e a tela baseados na caracterização feita no passo 9. \\
\subparagraph{11.} Avaliar a precisão do modelo utilizando medidas reais dos sistema com os benchmarks selecionados. \\
\subparagraph{12.} Explorar como padrões de utilização distintos podem afetar a precisão das estimativas do modelos de consumo de energia gerados. 
\subparagraph{13.} Verificar a generalidade da nossa metodologia aplicando a mesma para plataformas diferentes, como a plataforma ARM córtex-15. \\
\subparagraph{14.} Arquitetura ARM: investigar como coletar dados dos sensores de temperatura do processador. \\
\subparagraph{15.} Aprender a sincronizar e correlacionar medidas térmicas a pontos do código da aplicação. \\
\subparagraph{16.} Investigar como os modelos de consumo de energia gerados podem ser combinados com os dados térmicos para aplicar a “computational sprinting”.\\\\
\begin{tabular}{| l | l | l | l | l | l | l | l | l | l | l | l | l | l | l | l | l |}
  \hline
  mês/etapa & 01 & 02 & 03 & 04 & 05 & 06 & 07 & 08 & 09 & 10 & 11 & 12 & 13 & 14 & 15 & 16\\
  \hline
  1         & X & X &   &   &   &   &   &   &   &   &   &   &   &   &   & \\
  2         & X &   & X &   &   &   &   &   &   &   &   &   &   &   &   & \\
  3         & X &   & X & X &   &   &   &   &   &   &   &   &   &   &   & \\
  4         & X &   &   & X & X &   &   &   &   &   &   &   &   &   &   & \\
  5         &   &   &   &   &   & X &   &   &   &   &   &   &   &   &   & \\
  6         &   &   &   &   &   &   & X &   &   &   &   &   &   &   &   & \\
  7         &   &   &   &   &   &   & X &   &   &   &   &   &   &   &   & \\
  8         &   &   &   &   &   &   &   & X &   &   &   &   &   &   &   & \\
  9         &   &   &   &   &   &   &   & X & X &   &   &   &   &   &   & \\
  10        &   &   &   &   &   &   &   & X &   & X &   &   &   &   &   & \\
  11        &   &   &   &   &   &   &   &   &   & X &   &   &   &   &   & \\
  12        &   &   &   &   &   &   &   &   &   &   & X &   &   &   &   & \\
  13        &   &   &   &   &   &   &   &   &   &   &   & X &   &   &   & \\
  14        &   &   &   &   &   &   &   &   &   &   &   & X &   &   &   & \\
  15        &   &   &   &   &   &   &   &   &   &   &   &   & X &   &   & \\
  16        &   &   &   &   &   &   &   &   &   &   &   &   & X &   &   & \\
  17        &   &   &   &   &   &   &   &   &   &   &   &   & X &   &   & \\
  18        &   &   &   &   &   &   &   &   &   &   &   &   &   & X &   & \\
  19        &   &   &   &   &   &   &   &   &   &   &   &   &   &   & X & \\
  20        &   &   &   &   &   &   &   &   &   &   &   &   &   &   & X & \\
  21        &   &   &   &   &   &   &   &   &   &   &   &   &   &   & X & \\
  22        &   &   &   &   &   &   &   &   &   &   &   &   &   &   &   & X\\
  23        &   &   &   &   &   &   &   &   &   &   &   &   &   &   &   & X\\
  24        &   &   &   &   &   &   &   &   &   &   &   &   &   &   &   & X\\
  \hline
\end{tabular}

\bibliography{report}
\end{document}
